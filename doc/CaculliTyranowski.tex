\documentclass[11pt]{article}

\usepackage{amsmath}
\usepackage[french, onelanguage]{algorithm2e}
\usepackage[french]{babel}
\usepackage{geometry}
\usepackage[utf8]{inputenc}
\usepackage{listings}
\usepackage{lstautogobble}
\usepackage{tikz}
\usepackage{verbatim}
\usepackage{xcolor}

\usetikzlibrary{calc, shapes.multipart, chains, arrows}

\geometry{
  a4paper,
  total={170mm,257mm},
  left=20mm,
  top=20mm,
}

\definecolor{RoyalBlue}{cmyk}{1, 0.50, 0, 0}

\lstset{language=C,
  keywordstyle=\color{RoyalBlue},
  basicstyle=\scriptsize\ttfamily,
  commentstyle=\ttfamily\itshape\color{gray},
  stringstyle=\ttfamily,
  breaklines=true,
  keepspaces=true,
  numbers=left,
  numbersep=5pt,
  showspaces=false,
  showstringspaces=false,
  showtabs=false,
  tabsize=4,
  autogobble=true
}

\title{Documentation Mini Projet Langage C}
\author{Caculli Giorgio \& Jedrzej Tyranowski}
\date{\today}

\begin{document}

\maketitle
\begin{abstract}
\end{abstract}

\newpage
\section{Introduction}

\newpage
\section{Listes chaînées}

\newpage
\section{Énoncé}

\newpage
\section{Programme}
\subsection{Mode d'emploi}

\newpage
\section{Code}
\subsection{Structures}
\begin{lstlisting}
  typedef struct personne
  {
    int id;                      // L'ID unique de la personne
    char nom[25];                // Le nom de la personne
    char prenom[25];             // Le prenom de la personne
    int formateur;               
    int nb_formations;
    int formations[30];
    int nb_jours_indisponible;
    int jours_indisponible[7];
    int reduction;
    int val_reduction;
  } personne;
\end{lstlisting}

\begin{lstlisting}
  typedef struct noeud_db_personne
  {
    personne *p;
    struct noeud_db_personne *next;
  } noeud_db_personne;
\end{lstlisting}

\begin{lstlisting}
  typedef struct db_personne
  {
    noeud_db_personne *head;
  } db_personne;
\end{lstlisting}

\begin{lstlisting}
  typedef struct noeud_formation
  {
    personne *p;
    struct noeud_db_formation *next;
  } noeud_formation;
\end{lstlisting}

\begin{lstlisting}
  typedef struct formation
  {
    int id;
    char nom[40];
    float prix;
    int nb_jours;
    int jours[7];
    int heures[24];
    int durees[10];
    int nb_prerequis;
    int prerequis;
    noeud_formation *head;
  } formation;
\end{lstlisting}

\begin{lstlisting}
  typedef struct noeud_db_formation
  {
    formation *f;
    struct noeud_db_formation *next;
  } noeud_db_formation;
\end{lstlisting}

\begin{lstlisting}
  typedef struct db_formation
  {
    noeud_db_formation *head;
  } db_formation;
\end{lstlisting}

\subsection{Fonctions}

\newpage
\section{Structurogramme}
\begin{comment}
  \begin{algorithm}[H]
    \Begin{
        Ouvrir personne.dat en lecture\;
        Ouvrir formation.dat en lecture\;
        \While{!feof(personne.dat)}{
          lire nom, prenom, formateur nb\_formations\;
          \If{feof(personne.dat)}{
            break\;
          }
          ajouter\_db\_personne(dbp, creer\_personne(nom, prenom, formateur))\;
          get\_personne(dbp, nom, prenom, formateur)$\rightarrow$nb\_formations = nb\_formations\;
          get\_personne(dbp, nom, prenom, formateur)$\rightarrow$id = i + 1\;
          \For{$j=0$ \KwTo nb\_formations}{
            lire get\_personne(dbp, nom, prenom, formateur)$\rightarrow$formations[j]\;
          }
          i += 1\;
        }
        \While{!feof(formation.dat)}{
          lire id, prix, nom\_formation\;
          \If{feof(formation.dat)}{
            break\;
          }
          ajouter\_db\_formation(dbf, creer\_formation(nom\_formation, prix))\;
          get\_formation(dbf, nom\_formation)$\rightarrow$id = i + 1\;
          i += 1\;
        }
        tmpdbp = dbp\;
        tmpndbp = tmpdbp$\rightarrow$head\;
        \While{tmpndb != NULL}{
          tmpdbf = dbf\;
          tmpndbf = tmpdbf$\rightarrow$head\;
          \While{tmpndbf != NULL}{
            \For{$j=0$ \KwTo tmpndb$\rightarrow$p$\rightarrow$nb\_formations}{
              \If{tmpndbp$\rightarrow$p$\rightarrow$formations[j] = tmpndbf$\rightarrow$f$\rightarrow$id}{
                ajouter\_formation(get\_formation(dbf, tmpndbf$\rightarrow$f$\rightarrow$nom), tmpndbp$\rightarrow$p)\;
              }
              tmpndbf = tmpndbf$\rightarrow$next\;
            }
            tmpndbp = tmpndbp$\rightarrow$next\;
          }
        }
        menu(dbf, dbp)\;
      }
    \end{algorithm}
  \end{comment}

  \newpage
  \section{Listes chaînées}
  \begin{tikzpicture}[list/.style={
        rectangle split,
        rectangle split parts=2,
        draw,
        rectangle split horizontal
      },
      >=stealth,
      start chain]

    \node[list,on chain] (A) {12};
    \node[list,on chain] (B) {99};
    \node[list,on chain] (C) {37};
    \node[on chain,draw,inner sep=6pt] (D) {};
    \draw (D.north east) -- (D.south west);
    \draw (D.north west) -- (D.south east);
    \draw[*->] let \p1 = (A.two), \p2 = (A.center) in (\x1,\y2) -- (B);
    \draw[*->] let \p1 = (B.two), \p2 = (B.center) in (\x1,\y2) -- (C);
    \draw[*->] let \p1 = (C.two), \p2 = (C.center) in (\x1,\y2) -- (D);
  \end{tikzpicture}

  \newpage
  \begin{lstlisting}[language=C]
    typedef struct client
    {
      char nom[50];
      int age;
      float poids
    } client;
  \end{lstlisting}
  
\end{document}
