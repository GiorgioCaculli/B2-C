\documentclass[11pt]{article}

\usepackage{amsmath}
\usepackage[french, onelanguage]{algorithm2e}
\usepackage[french]{babel}
\usepackage{geometry}
\usepackage[utf8]{inputenc}
\usepackage{listings}
\usepackage{tikz}
\usepackage{xcolor}

\usetikzlibrary{calc, shapes.multipart, chains, arrows}

\geometry{
  a4paper,
  total={170mm,257mm},
  left=20mm,
  top=20mm,
}

\definecolor{codegreen}{rgb}{0,0.6,0}
\definecolor{codegray}{rgb}{0.5,0.5,0.5}
\definecolor{codepurple}{rgb}{0.58,0,0.82}
\definecolor{backcolour}{rgb}{0.95,0.95,0.92}

\lstdefinestyle{mystyle}{
    backgroundcolor=\color{backcolour},   
    commentstyle=\color{codegreen},
    keywordstyle=\color{magenta},
    numberstyle=\tiny\color{codegray},
    stringstyle=\color{codepurple},
    basicstyle=\ttfamily\footnotesize,
    breakatwhitespace=false,         
    breaklines=true,                 
    captionpos=b,                    
    keepspaces=false,                 
    numbers=left,                    
    numbersep=5pt,                  
    showspaces=false,                
    showstringspaces=false,
    showtabs=false,                  
    tabsize=4
}

\lstset{style=mystyle}

\title{Documentation Mini Projet Langage C}
\author{Caculli Giorgio \& Jedrzej Tyranowski}
\date{\today}

\begin{document}

\maketitle
\begin{abstract}
\end{abstract}

\newpage
\section{Structurogramme}
\begin{algorithm}[H]
  \Begin{
      Ouvrir personne.dat en lecture\;
      Ouvrir formation.dat en lecture\;
      \While{!feof(personne.dat)}{
        lire nom, prenom, formateur nb\_formations\;
        \If{feof(personne.dat)}{
          break\;
        }
        ajouter\_db\_personne(dbp, creer\_personne(nom, prenom, formateur))\;
        get\_personne(dbp, nom, prenom, formateur)$\rightarrow$nb\_formations = nb\_formations\;
        get\_personne(dbp, nom, prenom, formateur)$\rightarrow$id = i + 1\;
        \For{$j=0$ \KwTo nb\_formations}{
          lire get\_personne(dbp, nom, prenom, formateur)$\rightarrow$formations[j]\;
        }
        i += 1\;
      }
      \While{!feof(formation.dat)}{
        lire id, prix, nom\_formation\;
        \If{feof(formation.dat)}{
          break\;
        }
        ajouter\_db\_formation(dbf, creer\_formation(nom\_formation, prix))\;
        get\_formation(dbf, nom\_formation)$\rightarrow$id = i + 1\;
        i += 1\;
      }
      tmpdbp = dbp\;
      tmpndbp = tmpdbp$\rightarrow$head\;
      \While{tmpndb != NULL}{
        tmpdbf = dbf\;
        tmpndbf = tmpdbf$\rightarrow$head\;
        \While{tmpndbf != NULL}{
          \For{$j=0$ \KwTo tmpndb$\rightarrow$p$\rightarrow$nb\_formations}{
            \If{tmpndbp$\rightarrow$p$\rightarrow$formations[j] = tmpndbf$\rightarrow$f$\rightarrow$id}{
              ajouter\_formation(get\_formation(dbf, tmpndbf$\rightarrow$f$\rightarrow$nom), tmpndbp$\rightarrow$p)\;
            }
            tmpndbf = tmpndbf$\rightarrow$next\;
          }
          tmpndbp = tmpndbp$\rightarrow$next\;
        }
      }
      menu(dbf, dbp)\;
    }
  \end{algorithm}

  \newpage
  \section{Listes chaînées}
  \begin{tikzpicture}[list/.style={
        rectangle split,
        rectangle split parts=2,
        draw,
        rectangle split horizontal
      },
      >=stealth,
      start chain]

    \node[list,on chain] (A) {12};
    \node[list,on chain] (B) {99};
    \node[list,on chain] (C) {37};
    \node[on chain,draw,inner sep=6pt] (D) {};
    \draw (D.north east) -- (D.south west);
    \draw (D.north west) -- (D.south east);
    \draw[*->] let \p1 = (A.two), \p2 = (A.center) in (\x1,\y2) -- (B);
    \draw[*->] let \p1 = (B.two), \p2 = (B.center) in (\x1,\y2) -- (C);
    \draw[*->] let \p1 = (C.two), \p2 = (C.center) in (\x1,\y2) -- (D);
  \end{tikzpicture}

  \newpage
  \begin{lstlisting}[language=C]
    typedef struct client
    {
      char nom[50];
      int age;
      float poids
    } client;
  \end{lstlisting}
  
\end{document}
